\documentclass[a4paper, 11pt]{cours}

% XeTeX :
\XeTeXdefaultencoding utf-8
\usepackage{fontspec}
\usepackage{xunicode}
\usepackage{xltxtra}
\defaultfontfeatures{Mapping=tex-text}                 %% caractères spéciaux, comme ---
\setmainfont{Garamond Premier Pro}

\usepackage[frenchb]{babel}
%\usepackage[polutonikogreek,ngerman,english,frenchb]{babel}

\usepackage{array}
\usepackage[stable]{footmisc}     %% pour des footnotes dans les titres
\usepackage{pifont}               %% Dingbats

\usepackage{perpage}              %% remise à zéro des footnote à chaque nouvelle page
\MakePerPage{footnote}

% flowchart, graph et tout le reste
\usepackage{tikz}
\usetikzlibrary{shapes,arrows,calc}

% Pour les pdf interractifs :
% par défaut en noir, c'est mieux pour l'impression !
\usepackage
  [bookmarks=true,
  citecolor=black,
  filecolor=black,
  linkcolor=black,
  urlcolor=blue,
  unicode=false,             %% laisser en false sinon problème d'encodage avec acroread
  plainpages=false,
  pdfpagelabels,
  colorlinks=true,
  xetex]
  {hyperref} 

%%% insérer des images
%\usepackage[pdftex]{graphicx}
%\usepackage[small]{caption}


%% avoir des notes de bas de page à la française:
\FrenchFootnotes
\AddThinSpaceBeforeFootnotes

%% définition de l'environnement "squote" pour que les citations hors-texte soient dans une casse inférieure
\newenvironment{lquote}{\smallskip\begin{quote}\begin{small}\fontspec[Ligatures={Common, Rare}]{Garamond Premier Pro}}%
    {\end{small}\end{quote}\smallskip}

% idem avec quotation
\newenvironment{lquotation}{\smallskip\begin{quotation}\begin{small}\fontspec[Ligatures={Common, Rare}]{Garamond Premier Pro}}%
    {\end{small}\end{quotation}\smallskip}

%% Renommer des commandes
\renewcommand{\thesection}{\Roman{section}}	% Numérotation des chapitres en roman (I, II, ...)

% Le sublissime swash avec ses ligatures
\newcommand{\swash}[1]{{\fontspec[Ligatures={Common, Rare}]{Garamond Premier Pro Italic:+swsh}#1}}

% pour les numéros en OldStyle
\newcommand{\old}[1]{\fontspec[Numbers={OldStyle}]{Garamond Premier Pro}#1\fontspec[Numbers={Lining}]{Garamond Premier Pro}}

% commande des cours
\reversemarginpar
\newcommand{\cours}[1]{\marginpar{\fontspec[Ligatures={Common, Rare}, Numbers={OldStyle}]{Garamond Premier Pro}\textit{Cours #1.}}\vspace{-10pt}}

% les césure de mots non reconnus
% TODO : les insérer dans un fichier externe
\hyphenation{au-teur é-di-teur ho-ri-zon-tal im-pri-meur phy-si-que}